\documentclass[10pt]{beamer}

%% Chinese support
%% \usepackage[adobefonts,nocap]{ctex}

%% Fonts
\usepackage{multicol}
\usepackage{mathabx}
\usepackage[scaled]{helvet}
\usepackage{lmodern}
\usepackage{eulervm}
\usefonttheme[onlymath]{serif}
\usefonttheme{professionalfonts}
\usefonttheme{structurebold}
\usepackage{bm}
\usepackage{verbatim}

%% Color & Theme
\definecolor{SUblue}{RGB}{0,0,180}
\usecolortheme[RGB={0,0,180}]{structure}
\usetheme{Boadilla}
\setbeamertemplate{navigation symbols}{}
\setbeamertemplate{itemize items}[circle]
\setbeamertemplate{enumerate items}[circle]
\setbeamerfont{title}{size=\large}
\setbeamerfont{frametitle}{size=\large}
\setbeamerfont{framesubtitle}{size=\large,shape =$\color{violet}{\looparrowdownright}~$}
\setbeamercolor{title}{fg=white, bg= SUblue!75!green}
\setbeamercolor{framesubtitle}{fg=violet}
%\setlength{\leftmargini}{5pt}


\title[Statistical  Computing]{{\textbf{Loops and Functions}}}

\author[Feng Li]{\includegraphics[height=2cm]{cufelogo}\\
  \vspace{0.5cm}\textbf{Feng Li\\\texttt{feng.li@cufe.edu.cn}}}

\date{}
\institute[SAM.CUFE.EDU.CN]{\footnotesize{\textbf{School of Statistics and
      Mathematics\\ Central University of Finance and Economics}}}


%%%%%%%%%%%%%%%%%%%%%%%%%%%%%%%%%%%%%%%%%%%%%%%%%%%%%%%%%%%%%%%%%%%%%%
\begin{document}

%% Title page
\begin{frame}[plain]
  \titlepage
  \tiny{Revised on \today}
\end{frame}


%% Outline page
\section*{Today we are going to learn...}
\begin{frame}
  \frametitle{Today we are going to learn...}
  \tableofcontents
\end{frame}

\section{Control-flow constructs}

\begin{frame}[fragile]
\frametitle{The if condition}

\begin{itemize}
\item Comparison
\begin{verbatim}
x == y
x != y
x > y
x < y
x >= y
x <= y
all.equal()
%in%
\end{verbatim}

\item What would expect when $x$ and $y$ are vectors, matrices,...

\item The if condition

\begin{verbatim}
if (condition)
{
  do something
}
else
{
  do something else
}

\end{verbatim}

\end{itemize}

\end{frame}

\begin{frame}
\frametitle{Lab 1.3}

  % \section{}
\begin{enumerate}
\item February 29, known as a leap day in the calendar, is a date that occurs
  in most years that are evenly divisible by $4$, such as $2004$, $2008$,
  $2012$ and $2016$. Years that are evenly divisible by $100$ do not contain a
  leap day, with the exception of years that are evenly divisible by $400$,
  which do contain a leap day; thus $1900$ did not contain a leap day while
  $2000$ did.

\item Write a function called \texttt{is.leapday} to check if a given year has
  February 29 [Hint:  you may need \texttt{?\%\%}.].

\item Test your function for some years.

\item What can you do to improve for the function in terms of error tolerance?

\item If I want to check which year has a leap day for a sequence of given
  years. Modify your function to implement it.
\end{enumerate}

\end{frame}

\begin{frame}[fragile]
\begin{verbatim}
isLeapday <- function(year)
  {
   mod4 <- year%%4
   mod100 <- year%%100
   mod400 <- year%%400

   LeapdayIndex = ((mod4  == 0 & mod100 != 0) |  mod400  == 0)
   ## if((mod4  == 0 & mod100 != 0) |  mod400  == 0)
   ##   {
   ##     out <- TRUE
   ##   }
   ## else
   ##   {
   ##     out <- FALSE
   ##   }
   ## return(out)
   return(LeapdayIndex)
  }
\end{verbatim}
\end{frame}

\begin{frame}[fragile]
\begin{verbatim}
whichLeapday <- function(year)
  {
    if(!is.numeric(year))
      {
        stop("You must specify a numerical input.")
      }

   mod4 <- year%%4
   mod100 <- year%%100
   mod400 <- year%%400

   leapdayIndex <- ((mod4  == 0 & mod100 != 0) | mod400  == 0)

    out <- year[leapdayIndex]

    return(out)
  }
\end{verbatim}
\end{frame}


\section{Functions}
\begin{frame}[fragile]
  \frametitle{Functions}

  \begin{itemize}
  \item Create a function object
\begin{verbatim}
myFun = function (par)
{
  out = max(par1) - min(par2)
  return(out)
}

\end{verbatim}

    \item Load the function: \texttt{source()}

    \item Execute your function

  \end{itemize}

\end{frame}

\begin{frame}
  \frametitle{Functions}

  \begin{itemize}
  \item The input parameters type
  \item Validate the inputs
  \item Error catching
  \item Return types
  \end{itemize}
\end{frame}




\begin{frame}
\frametitle{Lab 1.1}
  \begin{enumerate}
\item Write a function \texttt{mySummary} where the input argument is
  \texttt{x} can be any vector and the output should contain the basic
  summary (mean, variance, length,
  max and minimum values, type) of the vector you have
  supplied to the function.
% \item What is the best way to construct your output?
\item Test your function with some vectors (that you make up by yourself).
\item What will happen if your input is not a vector (e.g. a data frame \texttt{weekPlanNew})
  in our previous example?
\end{enumerate}


\end{frame}

\begin{frame}
  %% \section{$\heartsuit$}
\frametitle{Lab 1.2}

\begin{enumerate}
\item The roots for the quadratic equation $ax^2+bx+c=0$ are of the form
      \begin{equation*}
        \label{eq:1}
        x_1=\frac{-b + \sqrt {b^2-4ac}}{2a} \quad \text{and} \quad
        x_2=\frac{-b - \sqrt {b^2-4ac}}{2a}
      \end{equation*}
\item Write a function named \texttt{quaroot} to solve the roots of given
  quadratic equation with \texttt{a ,b , c,} as input
  arguments. [Hint: you may need the \texttt{sqrt()} function]
\item Test your function on the following equations
  \begin{equation*}
    \label{eq:2}
    \begin{split}
      x^2+4x-1=0\\
      -2x^2+2x=0\\
      3x^2-9x+1=0\\
      x^2 -4 = 0\\
    \end{split}
  \end{equation*}
\item Test your function with the equation $5x^2+2x+1=0$. What are the results? Why? [Hint: check $b^2-4ac$]?
\item Modify your function and return \texttt{NA} if $b^2-4ac < 0$.
\end{enumerate}


\end{frame}


\begin{frame}[fragile]
\begin{verbatim}
quaroot <- function(a, b, c)
  {
    x1 <- (-b+sqrt(b^2-4*a*c))/(2*a)
    x2 <- (-b-sqrt(b^2-4*a*c))/(2*a)

    out <- c(x1, x2)
    return(out)
  }
\end{verbatim}
\end{frame}

\begin{frame}[fragile]
\begin{verbatim}
quaroot <- function(a, b, c)
  {
    d <- b^2-4*a*c

    if(d<0)
      {
        x1 <- NA
        x2 <- NA
      }
    else
      {
        x1 <- (-b+sqrt(d))/(2*a)
        x2 <- (-b-sqrt(d))/(2*a)
      }

    out <- c(x1, x2)
    return(out)
  }
\end{verbatim}
\end{frame}

\section{Loops}

\begin{frame}[fragile]
  \frametitle{The for loop}

\begin{verbatim}

B = matrix(1:10,2,5)
C = matrix(100:109,2,5)
A = matrix(NA,2,5)
for(i in 1:n)
{
  A[i] = B[i] + C[i]
}
\end{verbatim}
\end{frame}

\begin{frame}[fragile]
  \frametitle{The while loop}

\begin{verbatim}

B = matrix(1:10,2,5)
C = matrix(100:109,2,5)
A = matrix(NA,2,5)

i = 0
while(i != 10)
{  i = i + 1
  A[i] = B[i] + C[i]
}
\end{verbatim}
\end{frame}


\begin{frame}[allowframebreaks]
  \frametitle{\texttt{apply()} type loops}

  \begin{itemize}
  \item Calculate row sums for a matrix with a loop.
  \item Apply \texttt{sum()} function to each row of the matrix.

  \item \texttt{apply()} to an array with higher dimension.

  \item Apply your own function to each row of the matrix.
  \item \texttt{lapply()} Apply a function to a list

  \item \texttt{mapply()} Apply a function to multiple list or vector arguments.
  \item The ... arguments in a function.
  \item Supply more arguments to \texttt{apply()} type functions

  \item The advantage of \texttt{()apply}.

    \begin{itemize}
    \item Easy construct
    \item Less coding
    \item ()apply type loops is essentially a more efficient version loop.
    \end{itemize}
  \end{itemize}

\end{frame}

\begin{frame}[fragile]
  \frametitle{Write efficient loops}

  \begin{itemize}
  \item Avoid loops as much as possible.
  \item Use \texttt{()apply} type loop if possible.
  \item Think a lot about under- and over-flow
  \item Allocate the memory space before looping. This is a much
    slower loop.
\begin{verbatim}
B = matrix(1:10,2,5)
C = matrix(100:109,2,5)
A = NULL
for(i in 1:n)
{
  A[i] = B[i] + C[i]
}
\end{verbatim}

  \end{itemize}

\end{frame}




\begin{frame}
  \frametitle{Suggested reading}

  \begin{itemize}
  \item R-intro (2015): \textbf{Chapter 9, 10}
  \item Jones (2009): \textbf{Chapter 6.4}
  \end{itemize}

\end{frame}

\end{document}
