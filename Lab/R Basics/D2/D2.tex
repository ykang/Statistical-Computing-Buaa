\documentclass[a4paper]{article}
\usepackage[a4paper]{geometry}
\geometry{verbose,tmargin=2.5cm,bmargin=2.5cm,lmargin=2.5cm,rmargin=2.5cm}
\usepackage{amsmath}
\usepackage{fancyhdr}
%The first page setting
\fancypagestyle{plain}
{%
  \fancyhf{} % clear all header and footer fields
  \fancyhead[L]{
    Central University of Finance and Economics\\
    School of Statistics and Mathematics\\
    Feng Li
  }
  \fancyhead[R]{Programming in R}
}
%The remaining pages
\pagestyle{fancy}
\fancyhead[RO,LE]{}
\fancyhead[C]{Programming in R}
\fancyhead[LO,RE]{}


\title{D1--Introduction to Computers, Programming and R}
\date{\small{Revised on \today}}

\begin{document}
\maketitle
\hrule


\section{$\heartsuit$}
  \begin{enumerate}
   \item Use the \texttt{seq()} function to create the following sequences.
    \begin{enumerate}
    \item \texttt{1  2  3  4  5  6  7  8  9 10}
    \item \texttt{10  9  8  7  6  5  4  3  2  1}
    \item \texttt{1  3  5  7  9 11 13 15}
    \end{enumerate}
  \item Generate a sequence of length $12$ that starts from $0$ and ends up with
    $1$.
  \item  Generate a sequence that starts from $0$ and maximumly ends up with
    $1$ with the increment of $0.03$.
  \item Generate a sequence of length of $7$ that starts from $1$ with increment
    of $1.3$.
  \item Generate a sequence of length of $7$ that starts from $1$ with increment
    of $-1.3$.
  \item Generate a sequence of length $8$ that ends up with $20$ and with the
    increment of $2.2$.
  \item Generate a sequence of length $8$ that ends up with $20$ and with the
    increment of $-2.2$.
  \item Compute the mean, variance, standard deviation of the sequences you
    have generated from \textbf{1.6}.
  \item Use \texttt{rep()} function to generate the following sequences.
    \begin{enumerate}
    \item \texttt{"hello" "hello" "hello"}
    \item \texttt{TRUE TRUE TRUE TRUE TRUE}
    \item  \texttt{2 2 2 2 2 2 2 2 2 2}
    \end{enumerate}
  \item Use \texttt{seq()} or \texttt{rep()}, or both, to generate the following
    sequences and assign them to variables \texttt{s1.10a} and \texttt{s1.10b}
    respectively.
    \begin{enumerate}
    \item \texttt{1 2 3 4 5 1 2 3 4 5 1 2 3 4 5}
    \item \texttt{1  4  9 16 25 36 49 64 81}
    \end{enumerate}
  \item You are given a vector \texttt{w <- seq(5,10,1.2)}. Use \textbf{rep()}
    to generate a sequence that each element in \texttt{w} has been repeated
    three times, i.e.
\begin{verbatim}
5.0 5.0 5.0 6.2 6.2 6.2 7.4 7.4 7.4 8.6 8.6 8.6 9.8 9.8 9.8
\end{verbatim}
\end{enumerate}

\section{}
\begin{enumerate}
\item $\heartsuit$ Check if the elements in the sequence in question \textbf{1.10.(a)} satisfy following conditions.
  \begin{enumerate}
  \item The elements are greater than $2$.
  \item The elements are smaller than $3$.
  \item The elements are greater or equal than $2$.
  \item The elements are smaller or equal than $3$.
  \item The elements are equal to $5$.
  \item The elements are greater than $2$ but smaller than $4$.
  \item The elements are greater than $3$ or smaller than $2$.
  \end{enumerate}
\item Check the help of \texttt{which()} function and find out which elements in
  \textbf{1.10.(a)} satisfy the conditions in \textbf{2.1}.
\item  Select the elements form \textbf{1.10.(a)} that satisfies the
  condition in \textbf{2.1}.
\item $\heartsuit$ Consider the sequence in \textbf{1.10.(b)} and preform the following
  tasks.
  \begin{enumerate}
  \item Delete the first entry.
  \item Delete the last entry.
  \item Delete the last four elements.
  \item Select the entries that in the even positions.
  \item Select the entries that contains odd numbers[Hint: try
    \texttt{?\%\%}].
  \item Select the entries that is greater than the mean value of the
    sequence.
  \end{enumerate}
\end{enumerate}

\section{}
\begin{enumerate}
\item Consider the following sequence
\begin{verbatim}
> a <-  c(seq(1, 10, 2.2), rep(NA, 3), seq(10, 20, 1.5), rep(NaN, 4))
> a
 [1]  1.0  3.2  5.4  7.6  9.8   NA   NA   NA 10.0 11.5 13.0 14.5 16.0 17.5 19.0
[16]  NaN  NaN  NaN  NaN
\end{verbatim}
  and calculate the mean and variance of the sequence by using \texttt{mean(a)}
  and \texttt{var(a)}. What results do you find? Also check the type of the
  sequence.
\item Repeat the previous tasks in \textbf{3.1} but plug in an extra argument \texttt{na.rm =
    TRUE} in the \texttt{mean()} and \texttt{var()} functions. What results will
  you have now?
\item Use \texttt{is.na()} and \texttt{is.nan()} to check if the entries in the
  sequence are \textbf{not available} or \textbf{not a number}.
\item According to \textbf{3.3}, if you only want to remove entries with
  \texttt{NaN}, what command should you issue? If you only want to remove
  entries with \texttt{NA}, what command should you issue?
\item Remove \texttt{NA} and \texttt{NaN} in the sequence and calculate the
  mean and variance for the remaining elements. Compare your results with the
  results in \textbf{3.1} and \textbf{3.2}.
\end{enumerate}

\section{$\heartsuit$}
\begin{enumerate}
\item Assume you have a vector with the weekdays on which you need to study at the school,
\begin{verbatim}
> study <- c("Monday", "Tuesday", "Thursday")
> study
[1] "Monday"   "Tuesday"  "Thursday"
\end{verbatim}
but you also decided go to school on Wednesday. Insert \texttt{"Wednesday"} to your
\texttt{study} vector at proper location.
\item You have a part-time job at the student union on Friday, so you have
  \texttt{job <- "Friday"}. Also, let \texttt{fun <-
    c("Saturday","Sunday")}. Merge all "\texttt{study}", "\texttt{job}" and
  "\texttt{fun}" into a new vector called "\texttt{weekdays}"
\item You create a vector to indicate how much you are going to work or study
  during that week,
\begin{verbatim}
> hours <- c(rep(4, 4), 6, 0, 0)
> hours
[1] 4 4 4 4 6 0 0
\end{verbatim}

\item Your tasks for each day is then
\begin{verbatim}
> tasks <- c(rep("study", 4), "job", rep("fun", 2))
> tasks
[1] "study" "study" "study" "study" "job"   "fun"   "fun"
\end{verbatim}
  Now convert your vector \texttt{tasks} to factor to indicate the type of
  tasks and name it as \texttt{tasks}.

\item Check the class type of the four vectors you have now using \texttt{class()}.

\item Create a list with the vectors \texttt{weekdays}, \texttt{hours} and
  \texttt{tasks} and name it as \texttt{weekPlan} and print \texttt{weekPlan} on the R
  prompt.

\item Use the \texttt{names()} function to check if you have attributed the names
  to the elements of your list. If not, name each element with the vector's name so
  that your \texttt{weekPlan} will be something like this
\begin{verbatim}
> weekPlan
$weekdays
[1] "Monday"    "Tuesday"   "Wednesday" "Thursday"  "Friday"    "Saturday"
[7] "Sunday"

$hours
[1] 4 4 4 4 6 0 0

$tasks
[1] study study study study job   fun   fun
Levels: fun job study
\end{verbatim}

\item Perform some basic tasks to check the type and length of
  \texttt{weekPlan}. Check if there are missing values in the list. Check how
  much memory this object occupies [Hint \texttt{?object.size()}].  Try
  \texttt{str(weekPlan)} to display the internal structure of this object.

\item Append the new variable \texttt{note <- "plan for week 10"} to the current
  list \texttt{weekPlan}. How many ways can you find to do this append?

\item Extract \texttt{weekdays} from the list \texttt{weekPlan} using
  "\texttt{[[}" and "\texttt{\$}".

\item Select the first entry in the list \texttt{weekPlan}.

\item Create a new list named \texttt{weekPlan2} with entries \texttt{weekdays}
  and \texttt{tasks} from \texttt{weekPlan}.

\item Create a new list named \texttt{weekPlan3} with the first and second
  entries from \texttt{weekPlan}.

\item Delete \texttt{note} from the list \texttt{weekPlan}.

\item Find out on which days you have to work for more than 4 hours.

\end{enumerate}

\section{}
\begin{enumerate}
\item Create a data frame with the vectors \texttt{weekdays}, \texttt{hours},
  \texttt{tasks} that you obtained in \textbf{4.1--4.4} and name it as
  \texttt{weekPlanNew}
\begin{verbatim}
> weekPlanNew
   weekdays hours tasks
1    Monday     4 study
2   Tuesday     4 study
3 Wednesday     4 study
4  Thursday     4 study
5    Friday     6   job
6  Saturday     0   fun
7    Sunday     0   fun
\end{verbatim}

\item Perform the following basis tasks for the data frame \texttt{weekPlanNew}
  \begin{enumerate}
  \item Check the structure of the data frame.
  \item Check the number of rows and columns in the data frame.
  \item What is the dimension of the data frame?
  \item Check the row and column names of the data frame.
  \end{enumerate}

\item Create a new vector called \texttt{costs} for your daily cost in
  SEK. \texttt{costs <- c(70, 75, 58, 62, 140, 90, 70)} and append it to the
  end of the data frame.

\item $\heartsuit$ Answer the following questions for your new \texttt{weekPlanNew}
  \begin{enumerate}
  \item What are you going to do on the first day of the week? How long will
    you work on that day?
  \item What are you going to do for the last three days?
  \item How much money do you spend during the weekend in total, how about the
    whole week?
  \item How many hours do you work (not including study times) in total for this week?
  \item Which days do you work or study less than five hours?
  \item Which day do you spend more than 100SEK?
  \item You decide to go for work on Tuesday instead of studying. Change the
    corresponding task to job.
  \item Extract the first column from the data frame, and check the dimension of
    that column. Now extract the first and third columns of the data frame and then
    check the dimension of that newly obtained data frame.
  \item Repeat \textbf{5.4.(h)} but insert \texttt{drop = FALSE} argument when
    you subtract the columns.
  \end{enumerate}

\item You think the next week you will do the similar plan as this week, so you
  copy this week's plan and append it after the last row of current
  \texttt{weekPlanNew}.
\item You feel you went over budget this week, so you decide not to spend more than
  the smallest daily spending during this week. Modify the
  corresponding entries.
\item Discussion: when should you use data frame or list?
\end{enumerate}

\section{$\heartsuit$}
\begin{enumerate}
\item Write a function \texttt{mySummary} where the input argument is
  \texttt{x} can be any vector and the output should contain the basic
  summary (mean, variance, length,
  max and minimum values, type) of the vector you have
  supplied to the function.
% \item What is the best way to construct your output?
\item Test your function with some vectors (that you make up by yourself).
\item What will happen if your input is not a vector (e.g. a data frame \texttt{weekPlanNew})
  in our previous example?
\end{enumerate}

\section{$\heartsuit$}
\begin{enumerate}
\item The roots for the quadratic equation $ax^2+bx+c=0$ are of the form
      \begin{equation*}
        \label{eq:1}
        x_1=\frac{-b + \sqrt {b^2-4ac}}{2a} \quad \text{and} \quad
        x_2=\frac{-b - \sqrt {b^2-4ac}}{2a}
      \end{equation*}
\item Write a function named \texttt{quaroot} to solve the roots of given
  quadratic equation with \texttt{a ,b , c,} as input
  arguments. [Hint: you may need the \texttt{sqrt()} function]
\item Test your function on the following equations
  \begin{equation*}
    \label{eq:2}
    \begin{split}
      x^2+4x-1=0\\
      -2x^2+2x=0\\
      3x^2-9x+1=0\\
      x^2 -4 = 0\\
    \end{split}
  \end{equation*}
\item Test your function with the equation $5x^2+2x+1=0$. What are the results? Why? [Hint: check $b^2-4ac$]?
\item Modify your function and return \texttt{NA} if $b^2-4ac < 0$.
\end{enumerate}

% \section{}
% \begin{enumerate}
% \item February 29, known as a leap day in the calendar, is a date that occurs
%   in most years that are evenly divisible by $4$, such as $2004$, $2008$,
%   $2012$ and $2016$. Years that are evenly divisible by $100$ do not contain a
%   leap day, with the exception of years that are evenly divisible by $400$,
%   which do contain a leap day; thus $1900$ did not contain a leap day while
%   $2000$ did.

% \item Write a function called \texttt{is.leapday} to check if a given year has
%   February 29 [Hint:  you may need \texttt{?\%\%}.].

% \item Test your function for some years.

% \item What can you do to improve for the function in terms of error tolerance?

% \item If I want to check which year has a leap day for a sequence of given
%   years. Modify your function to implement it.
% \end{enumerate}

\end{document}
