\documentclass[10pt]{beamer}

%% Chinese support
%% \usepackage[adobefonts,nocap]{ctex}

%% Fonts
\usepackage{multicol}
\usepackage{mathabx}
\usepackage[scaled]{helvet}
\usepackage{lmodern}
\usepackage{eulervm}
\usefonttheme[onlymath]{serif}
\usefonttheme{professionalfonts}
\usefonttheme{structurebold}
\usepackage{bm}
\usepackage{verbatim}

%% Color & Theme
\definecolor{SUblue}{RGB}{0,0,180}
\usecolortheme[RGB={0,0,180}]{structure}
\usetheme{Boadilla}
\setbeamertemplate{navigation symbols}{}
\setbeamertemplate{itemize items}[circle]
\setbeamertemplate{enumerate items}[circle]
\setbeamerfont{title}{size=\large}
\setbeamerfont{frametitle}{size=\large}
\setbeamerfont{framesubtitle}{size=\large,shape =$\color{violet}{\looparrowdownright}~$}
\setbeamercolor{title}{fg=white, bg= SUblue!75!green}
\setbeamercolor{framesubtitle}{fg=violet}
%\setlength{\leftmargini}{5pt}


\title[Statistical Computing]{{\textbf{Debugging and profiling in R}}}

\author[Feng Li]{\includegraphics[height=2cm]{cufelogo}\\
  \vspace{0.5cm}\textbf{Feng Li\\\texttt{feng.li@cufe.edu.cn}}}

\institute[Stat \& Math, CUFE]{\footnotesize{\textbf{School of Statistics and
      Mathematics\\ Central University of Finance and Economics}}}


%%%%%%%%%%%%%%%%%%%%%%%%%%%%%%%%%%%%%%%%%%%%%%%%%%%%%%%%%%%%%%%%%%%%%%%%%%%%%%%
\begin{document}

%% Title page
\begin{frame}[plain]
  \titlepage
  \tiny{Revised on \today}
\end{frame}


%% Outline page
% \section*{Today we are going to learn...}
% \begin{frame}
%   \frametitle{Today we are going to learn...}
%   \tableofcontents
% \end{frame}

\section{Debugging}

\begin{frame}
\frametitle{The basic concepts of debugging}

\begin{itemize}
\item \textbf{Debugging} is a methodical process of finding and reducing the number of bugs, or defects, in a computer program or a piece of electronic hardware, thus making it behave as expected.

\item A \textbf{debugger} or debugging tool is a computer program that
  is used to test and debug other programs (the "target" program)

\item Debugging involves numerous aspects including \textbf{interactive debugging}, \textbf{control flow}, \textbf{integration testing}, \textbf{log files}, \textbf{monitoring} (application, system), \textbf{memory dumps}, \textbf{profiling}, \textbf{Statistical Process Control}, and special design tactics to improve detection while simplifying changes.


\end{itemize}

\end{frame}

\begin{frame}
  \frametitle{Typical debugging process}

  \begin{itemize}
  \item Normally the first step in debugging is to attempt to
    reproduce the problem.

  \item After the bug is reproduced, the input of the program may need
    to be simplified to make it easier to debug

  \item After the test case is sufficiently simplified, a programmer
    can use a debugger tool to examine program states (values of
    variables, plus the call stack) and track down the origin of the
    problem(s). Alternatively, tracing can be used

  \end{itemize}
\end{frame}

\begin{frame}
  \frametitle{Debugging tools in R}

  \begin{itemize}
  \item The simplest version:

    \texttt{cat(); print()}

  \item \texttt{browser()}

    At the browser prompt the user can enter commands or R
    expressions, followed by a newline.  The commands are

    \begin{itemize}
    \item 'c' (or just an empty line, by default) exit the browser and
      continue execution at the next statement.

    \item 'n' enter the step-through debugger if the function is
      interpreted.  This changes the meaning of 'c': see the
      documentation for 'debug'. For byte compiled functions 'n' is
      equivalent to 'c'.

    \item 'where' print a stack trace of all active function calls.

    \item 'Q' exit the browser and the current evaluation and return to
      the top-level prompt.

    \item \texttt{> options(browserNLdisabled = TRUE)}


    \end{itemize}

  \item \texttt{trace()} \texttt{traceback()}

  \item if control flow

  \item \texttt{ls()}

  \item \texttt{try()} \texttt{tryCatch()}

  \end{itemize}
\end{frame}


\section{Profiling your code}

\begin{frame}
  \frametitle{Why profiling?}

  \begin{itemize}
  \item Find the computational bottom-neck of your code.
  \item Fine the memory bottom-neck of your code.
  \end{itemize}

\end{frame}


\begin{frame}[fragile]
  \frametitle{Profiling R code for speed}

  \begin{itemize}
  \item Check computing time of a piece of code: \texttt{proc.time()}.
  \item Profiling works by recording at fixed intervals (by default
    every 20 msecs) which line in which R function is being used, and
    recording the results in a file.
  \item The R profiling procedure

\begin{verbatim}
Rprof("myprofile.out") # Open the profile log file
##
.... ## Some code you want to profile
##
Rprof(NULL) # Close the profile log

summaryRprof("myprofile.out") # summarize the results

\end{verbatim}
  \end{itemize}

\end{frame}


\begin{frame}[fragile,allowframebreaks]
  \frametitle{Profiling R code for memory use}

  \begin{itemize}
  \item Measuring memory use in R code is useful either when the code
    takes more memory than is conveniently available or when memory
    allocation and copying of objects is responsible for slow code.

  \item Garbage collection: gc()

\begin{verbatim}
>gc()
         used (Mb) gc trigger (Mb) max used (Mb)
Ncells 311043 16.7     597831 32.0   597831 32.0
Vcells 761909  5.9    1445757 11.1  1137162  8.7
\end{verbatim}

  \item \textbf{Vcells} used to store the contents of vectors
  \item \textbf{Ncells} used to store everything else, including all the
    administrative overhead for vectors such as type and length
    information. In fact the vector contents are divided into two
    pools.

\item The sampling profiler Rprof described in the previous section can be given the option
\texttt{memory.profiling=TRUE}.
\begin{verbatim}
Rprof("myprofile.out", memory.profiling=TRUE) # Open the profile log file
##
.... ## Some code you want to profile
##
Rprof(NULL) # Close the profile log

summaryRprof("myprofile.out") # summarize the results

\end{verbatim}

\item Memory profiling requires R to have been compiled with
  \texttt{--enable-memory-profiling}, which is not the default, but is
  currently used for the OS X and Windows binary distributions.

  \end{itemize}

\end{frame}

\begin{frame}
  \frametitle{Suggested Reading}

  \begin{itemize}
  \item Jones (2009), \textbf{Chapter 3.7, 5.6, 8.3, 9.3, 9.5}


  % \emph{Monte Carlo Statistical Methods} Book by Christian P Robert and George
  % Casella. (2004 edition)

  \end{itemize}

\end{frame}

\end{document}
