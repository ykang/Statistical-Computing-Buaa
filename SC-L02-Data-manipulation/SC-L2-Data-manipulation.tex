\documentclass[10pt]{beamer}

%% Chinese support
%% \usepackage[adobefonts,nocap]{ctex}

%% Fonts
\usepackage{multicol}
\usepackage{mathabx}
\usepackage[scaled]{helvet}
\usepackage{lmodern}
\usepackage{eulervm}
\usefonttheme[onlymath]{serif}
\usefonttheme{professionalfonts}
\usefonttheme{structurebold}
\usepackage{bm}
\usepackage{verbatim}

%% Color & Theme
\definecolor{SUblue}{RGB}{0,0,180}
\usecolortheme[RGB={0,0,180}]{structure}
\usetheme{Boadilla}
\setbeamertemplate{navigation symbols}{}
\setbeamertemplate{itemize items}[circle]
\setbeamertemplate{enumerate items}[circle]
\setbeamerfont{title}{size=\large}
\setbeamerfont{frametitle}{size=\large}
\setbeamerfont{framesubtitle}{size=\large,shape =$\color{violet}{\looparrowdownright}~$}
\setbeamercolor{title}{fg=white, bg= SUblue!75!green}
\setbeamercolor{framesubtitle}{fg=violet}
%\setlength{\leftmargini}{5pt}


\title[Statistical Computing]{{\textbf{Data manipulation}}}

\author[Feng Li]{\includegraphics[height=2cm]{cufelogo}\\
  \vspace{0.5cm}\textbf{Feng Li\\\texttt{feng.li@cufe.edu.cn}}}
\date{}
\institute[Stat \& Math, CUFE]{\footnotesize{\textbf{School of Statistics and
      Mathematics\\ Central University of Finance and Economics}}}


%%%%%%%%%%%%%%%%%%%%%%%%%%%%%%%%%%%%%%%%%%%%%%%%%%%%%%%%%%%%%%%%%%%%%%
\begin{document}

%% Title page
\begin{frame}[plain]
  \titlepage
  \tiny{Revised on \today}
\end{frame}


%% Outline page
\section*{Today we are going to learn...}
\begin{frame}
  \frametitle{Today we are going to learn...}
  \tableofcontents
\end{frame}


\section{Generate a sequence}
\begin{frame}
\frametitle{Sequences}

\begin{itemize}
\item Generate a sequence: \texttt{seq()}
\item Repeat a vector: \texttt{rep()}
\end{itemize}
\end{frame}

\begin{frame}
  \frametitle{Vectors}
  \begin{itemize}
  \item Numerical vectors
  \item Logical vectors
  \item Characters
  \item Length of a vector
  \item Vector calculations
  \end{itemize}
\end{frame}


\begin{frame}
  \frametitle{Mathematical functions}

  \begin{itemize}

  \item \texttt{sqrt(), log()}

  \item \texttt{sin(),cos(), tan()}


  \end{itemize}
\end{frame}


\section{Vectors and  matrices}




\begin{frame}
\frametitle{Matrices}
  \begin{itemize}
  \item Create a matrix: \texttt{matrix()}
  \item Dimension of a matrix: \texttt{dim()}
  \item How many elements in a matrix: \texttt{length()}
  \item Extract elements from a matrix.
  \item Replace elements with new entries.
  \item Create special matrices: diagonal matrix, identity matrix,
    zero matrix...
  \item Matrix multiplications: \texttt{\%*\%}
  \item Matrix inverse: \texttt{solve()}
  \item Transpose of a matrix: \texttt{t()}
  \item Element-wise operation with a matrix.
  \item Combine two or more matrices: \texttt{rbind(), cbind()}
  \end{itemize}
\end{frame}



\section{Vectorization}

\begin{frame}[fragile]
  \frametitle{The concept of vectorization}
  \begin{itemize}
  \item Vectorization: same operation applies to each element of an object.
  \item It is not a loop.

  \item   If you want to do four additions
    \begin{align*} c_1 & = a_1 + b_1 \\ c_2 & = a_2 + b_2 \\ c_3 & = a_3 + b_3
      \\ c_4 & = a_4 + b_4 \end{align*}
    in C you have to write it as
\begin{verbatim}
for (i=0; i<4; i++)
    c[i] = a[i] + b[i];
\end{verbatim}

  \item There is a vectorizing compiler in C that transform such a
    loop into a sequence of vector operations, that perform additions
    on length-four (in our example) blocks of elements from the arrays
    a, b and c when you compile you C code.

  \item In R, you can just do it as

\begin{verbatim}
c = a + b
\end{verbatim}

and R will do the rest for you.

  \end{itemize}
\end{frame}


\begin{frame}
  \frametitle{Vectorization with vectors and matrices}

  \begin{itemize}
  \item Vectorize your code as much as possible.

  \item Vectorization works with vectors, matrices and arrays.

  \item Special case

    \begin{itemize}
    \item When matrix multiplies (*) a scaler or a vector, it will
      first repeat the vector to be the same length of the
      matrix. Then do the element-wise multiplication.

    \item Same rule applies to other types of arithmetic.

    \end{itemize}

  \end{itemize}
\end{frame}

\section{Data Structures}

\begin{frame}
  \frametitle{Array}

  \begin{itemize}
  \item An array is a high dimensional matrix.
  \item A matrix is a special case of an array when the dimension is
    two.
  \item A vector is a special array when their is no dimension (In R
    the dimension is usually dropped in this situation)
  \end{itemize}
\end{frame}

\begin{frame}
  \frametitle{List}

  \begin{itemize}
  \item Special data structure that matrix could not handle.

    \begin{itemize}
    \item Data length are not the same.
    \item Data type are not the same.
    \item Nested data structure within a list.
    \end{itemize}

  \item Create a list: \texttt{list()}
  \item Extract elements of a \texttt{list: [[]]} or \texttt{\$}

  \item Delete an element within a list: set \texttt{NULL} to that element.

  \end{itemize}
\end{frame}


\begin{frame}
  \frametitle{Data frame}

  \begin{itemize}
  \item \texttt{data.frame()}: tightly coupled collections of
    variables which share many of the properties of matrices and of
    lists, used as the fundamental data structure by most of R's
    modeling software.

  \item In most cases, the operation with a data frame is similar to
    matrix operation.

  \end{itemize}

\end{frame}

\begin{frame}
\frametitle{Discussion}

  What type of data structure would you choose when you meet the
    following situations.

    \begin{itemize}
    \item Data are of the same length but different types.
    \item Data are not of the same length.
    \item Hierarchical structure of the data.
    \end{itemize}

\end{frame}


\begin{frame}
  \frametitle{Suggested reading}

  \begin{itemize}
  \item R-intro (2015): \textbf{Chapter 2, 3, 5, 6}
  \end{itemize}

\end{frame}

\end{document}
