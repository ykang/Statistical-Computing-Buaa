\documentclass[10pt]{beamer}

%% Chinese support
%% \usepackage[adobefonts,nocap]{ctex}

%% Fonts
\usepackage{multicol}
\usepackage{mathabx}
\usepackage[scaled]{helvet}
\usepackage{lmodern}
\usepackage{eulervm}
\usefonttheme[onlymath]{serif}
\usefonttheme{professionalfonts}
\usefonttheme{structurebold}
\usepackage{bm}
\usepackage{verbatim}

%% Color & Theme
\definecolor{SUblue}{RGB}{0,0,180}
\usecolortheme[RGB={0,0,180}]{structure}
\usetheme{Boadilla}
\setbeamertemplate{navigation symbols}{}
\setbeamertemplate{itemize items}[circle]
\setbeamertemplate{enumerate items}[circle]
\setbeamerfont{title}{size=\large}
\setbeamerfont{frametitle}{size=\large}
\setbeamerfont{framesubtitle}{size=\large,shape =$\color{violet}{\looparrowdownright}~$}
\setbeamercolor{title}{fg=white, bg= SUblue!75!green}
\setbeamercolor{framesubtitle}{fg=violet}
%\setlength{\leftmargini}{5pt}


\title[Statistical Computing]{{\textbf{Data Import and Export}}}

\author[Feng Li]{\includegraphics[height=2cm]{cufelogo}\\
  \vspace{0.5cm}\textbf{Feng Li\\\texttt{feng.li@cufe.edu.cn}}}

\institute[SAM.CUFE.EDU.CN]{\footnotesize{\textbf{School of
      Statistics and Mathematics\\ Central University of Finance and
      Economics}}}
\date{}
%%%%%%%%%%%%%%%%%%%%%%%%%%%%%%%%%%%%%%%%%%%%%%%%%%%%%%%%%%%%%%%%%%%%%%
\begin{document}

%% Title page
\begin{frame}[plain]
  \titlepage
  \tiny{Revised on \today}
\end{frame}


%% Outline page
\section*{Today we are going to learn...}
\begin{frame}
  \frametitle{Today we are going to learn...}
  \tableofcontents
\end{frame}

\section{Data import}

\begin{frame}
  \frametitle{Plain Text Format}

  \begin{itemize}
  \item Reads a file in table format and creates a data frame from it.
    \begin{itemize}
    \item \texttt{read.table()}
    \item \texttt{read.csv()}
    \item \texttt{read.csv2()}
    \end{itemize}

  \item Write a file in text format.
    \begin{itemize}
    \item \texttt{write.table()}
    \item \texttt{write.csv()}
    \item \texttt{write.csv2()}

    \end{itemize}
  \end{itemize}
\end{frame}

\begin{frame}[fragile]
  \frametitle{Micro\$oft Excel}

  \begin{itemize}

\item  Install and load the package
\begin{verbatim}
install.packages("XLConnect")
library("XLConnect")
\end{verbatim}

\item Read XLS Files

\begin{verbatim}
require("XLConnect")
wb = loadWorkbook("XLConnectExample1.xlsx", create = TRUE)
data = readWorksheet(wb, sheet = "chickSheet",
                     startRow = 0, endRow = 10,
                     startCol = 0, endCol = 0)
\end{verbatim}
  \end{itemize}
\end{frame}

\begin{frame}
  \frametitle{Minitab, S-PLUS, SAS, SPSS, Stata, Systat}

  The recommended package foreign provides import facilities for files
  produced by these statistical systems.

  \begin{itemize}
  \item Function \texttt{read.xport()} reads a file in SAS Transport
    (XPORT) format and return a list of data frames.

  \item Function \texttt{read.mtp()} imports a 'Minitab Portable
    Worksheet'. This returns the components of the worksheet as an R
    list

  \item Function \texttt{read.spss()} can read files created by the
    'save' and 'export' commands in SPSS


  \item Files from Stata can be read and written by functions
    \texttt{read.dta()} and \texttt{write.dta()}.

  \end{itemize}
\end{frame}


\begin{frame}
  \frametitle{Read and Write Matlab format}

  \textbf{R.matlab} package: Read and Write MAT Files and Call MATLAB from Within R (No
  Matlab installation required.)

  \begin{itemize}
  \item Methods \texttt{readMat()} and \texttt{writeMat()} for reading and writing MAT
    files. For user with MATLAB v6 or newer installed (either locally or on a remote
    host), the package also provides methods for controlling MATLAB (trademark) via R and
    sending and retrieving data between R and MATLAB.

  \end{itemize}
\end{frame}

\section{Read Database}

\begin{frame}[allowframebreaks,fragile]
  \frametitle{R and Database}
  \begin{itemize}

  \item SQL has limited numerical and statistical features. For example, it has no least
    squares fitting procedures, and to find quantiles requires a sophisticated query.

  \item Not only are basic statistical functions missing from SQL, but in many cases the
    numerical algorithms used in the basic aggregate functions are not implemented to
    safeguard numerical accuracy.

  \item For these reasons, it may be desireable or even necessary to perform a statistical
    analysis in a statistical package rather than in the database. One way to do this, is
    to extract the data from the database and import it into statistical software.

  \item The \texttt{RODBC} package provides access to databases (including Microsoft
    Access and Microsoft SQL Server) through an ODBC interface.

    \begin{itemize}
    \item \texttt{odbcConnect(dsn, uid="", pwd="")}

      Open a connection to an ODBC database

    \item \texttt{sqlFetch(channel, sqtable)}

      Read a table from an ODBC database into a data frame

    \item \texttt{sqlQuery(channel, query)}

      Submit a query to an ODBC database and return the results

    \item \texttt{sqlSave(channel, mydf, tablename = sqtable, append = FALSE)}

      Write or update (append=True) a data frame to a table in the ODBC database

    \item \texttt{sqlDrop(channel, sqtable)}

      Remove a table from the ODBC database close(channel) Close the connection

    \end{itemize}

  \item Use RODBC to read Oracle
    Database \texttt{https://blogs.oracle.com/R/entry/r\_to\_oracle\_database\_connectivity}

  \item The \textbf{DBI} package in R provides a uniform, client- side interface to
    different database management systems, such as MySQL, PostgreSQL, and Oracle. The
    basic model breaks the interface between the client and the server into three main
    elements: the driver facilitates the communication between the R session and a
    particular type of database management system (e.g. MySQL); the connection
    encapsulates the actual connection (with the aid of the driver) to a particular
    database management system and carries out the requested queries; and the result which
    tracks the status of a query, such as the number of rows that have been fetched and
    whether or not the query has completed

    \begin{itemize}
    \item We provide a simple example here of how to extract data from a MySQL database in
      an R session. The first step: load a driver for a MySQL-type database

\begin{verbatim}
      drv = dbDriver("MySQL")
\end{verbatim}

    \item The next step is to make a connection to the database management server of
      interes
\begin{verbatim}
con = dbConnect(drv, user="yourusername", dbname="DataBaseName",
                host="HostName")
\end{verbatim}

\item Once the connection is established, queries can be sent to the database

\begin{verbatim}
dbListTables(con)
\end{verbatim}

\item Other queries can be executed by supplying the SQL statement.

\begin{verbatim}
dbGetQuery(con,"SELECT COUNT(*) FROM Allstar;")
\end{verbatim}

    \end{itemize}


  \item The \textbf{RMySQL} package provides an interface to MySQL.

  \item The \textbf{ROracle} package provides an interface for Oracle.

  \item The \textbf{RJDBC} package provides access to databases through a JDBC interface.

  \end{itemize}
\end{frame}


\begin{frame}
  \frametitle{Lab exercise}

    \begin{itemize}
    \item Read and query data from database.
    \end{itemize}
  \end{frame}

\section{Read images}

\begin{frame}[fragile]
  \frametitle{Images}

  \begin{itemize}
  \item JPEG Format
\begin{verbatim}
install.packages("jpeg")
require("jpeg")
readJPEG()
rasterImage()
\end{verbatim}

  \item Packages \textbf{bmp}, \textbf{jpeg} and \textbf{png} read the
    formats after which they are named. See also packages
    \textbf{biOps} and \textbf{Momocs}, and Bioconductor package
    \textbf{EBImage}.

  \end{itemize}

\end{frame}


\begin{frame}[fragile]
  \frametitle{Images: Example}

\begin{verbatim}
library("jpeg")
myprofile <- readJPEG("FengLi.jpg")
dim(myprofile)
image(1:1539, 1:1154, myprofile[, , 1])
image(1:1539, 1:1154, myprofile[, , 2])
image(1:1539, 1:1154, myprofile[, , 3])

plot(c(100, 250), c(300, 450), type = "n", xlab = "", ylab = "")
rasterImage(myprofile, 100, 300, 150, 350, interpolate = FALSE)
rasterImage(myprofile, 100, 400, 150, 450)
rasterImage(myprofile, 200, 300, 200 + xinch(.5),
            300 + yinch(.3), interpolate = FALSE)
rasterImage(myprofile, 200, 400, 250, 450,
            angle = 15, interpolate = FALSE)
\end{verbatim}

\end{frame}


\begin{frame}
  \frametitle{Further Suggested Read}
  \begin{itemize}
  \item R Data Import/Export

    \url{http://cran.r-project.org/doc/manuals/r-release/R-data.pdf}

  \end{itemize}
\end{frame}

\end{document}
