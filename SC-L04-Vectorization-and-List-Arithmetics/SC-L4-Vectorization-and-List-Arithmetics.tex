\documentclass[10pt]{beamer}

%% Chinese support
%% \usepackage[adobefonts,nocap]{ctex}

%% Fonts
\usepackage{multicol}
\usepackage{mathabx}
\usepackage[scaled]{helvet}
\usepackage{lmodern}
\usepackage{eulervm}
\usefonttheme[onlymath]{serif}
\usefonttheme{professionalfonts}
\usefonttheme{structurebold}
\usepackage{bm}
\usepackage{verbatim}

%% Color & Theme
\definecolor{SUblue}{RGB}{0,0,180}
\usecolortheme[RGB={0,0,180}]{structure}
\usetheme{Boadilla}
\setbeamertemplate{navigation symbols}{}
\setbeamertemplate{itemize items}[circle]
\setbeamertemplate{enumerate items}[circle]
\setbeamerfont{title}{size=\large}
\setbeamerfont{frametitle}{size=\large}
\setbeamerfont{framesubtitle}{size=\large,shape =$\color{violet}{\looparrowdownright}~$}
\setbeamercolor{title}{fg=white, bg= SUblue!75!green}
\setbeamercolor{framesubtitle}{fg=violet}
%\setlength{\leftmargini}{5pt}


\title[Statistical Computing]{{\textbf{Vectorization and List Arithmetics}}}

\author[Feng Li]{\includegraphics[height=2cm]{cufelogo}\\
  \vspace{0.5cm}\textbf{Feng Li\\\texttt{feng.li@cufe.edu.cn}}}

\institute[Stat \& Math, CUFE]{\footnotesize{\textbf{School of
      Statistics and Mathematics\\ Central University of Finance and
      Economics}}}
\date{}

%%%%%%%%%%%%%%%%%%%%%%%%%%%%%%%%%%%%%%%%%%%%%%%%%%%%%%%%%%%%%%%%%%%%%%
\begin{document}

%% Title page
\begin{frame}[plain]
  \titlepage
  \tiny{Revised on \today}
\end{frame}


%% Outline page
\section*{Today we are going to learn...}
\begin{frame}
  \frametitle{Today we are going to learn...}
  \tableofcontents
\end{frame}

\section{Vectorization}

\begin{frame}
  \frametitle{Vectorization}

  \begin{itemize}
  \item The traditional \texttt{for} and \texttt{while} loops
  \item The concepts of vectorization
  \item Avoid loops by using vectorization
  \item What can be vectorized?
  \end{itemize}
\end{frame}


\begin{frame}
  \frametitle{Lab exercises}


  \begin{itemize}
  \item Compare the efficiency of a ``for'' loop and vectorization for calculating
    element-wise matrix multiplication.

  \item Hint: \texttt{proc.time()}, \texttt{replicate()}
  \end{itemize}

\end{frame}

\section{Apply a function to some margins of a matrix or an array}

\begin{frame}
  \frametitle{\textbf{Apply} a function to some margins of a matrix or an array}
  \begin{itemize}
  \item \texttt{apply(X, MARGIN, FUN, ...)}
  \end{itemize}
\end{frame}


\section{List Arithmetics}

\begin{frame}
  \frametitle{List Arithmetics}
  \framesubtitle{Apply a function to the elements of a list}
  \begin{itemize}
  \item \texttt{lapply(X, FUN, ...)}
  \item \texttt{rapply(object, f,  how = c("unlist","replace", "list"), ...)}
  \end{itemize}
\end{frame}


\begin{frame}
  \frametitle{List Arithmetics}
  \framesubtitle{Operators with many lists}
  \begin{itemize}
  \item  \texttt{mapply(FUN, ..., MoreArgs = NULL, SIMPLIFY = TRUE, USE.NAMES = TRUE)}

    \begin{itemize}
    \item \texttt{mapply("+", list1, list2, list3, SIMPLIFY = FALSE)}
    \item \texttt{mapply(function(x, y) abs(x)*log(abs(y)), list1, list2, SIMPLIFY = FALSE)}

    \end{itemize}

  \end{itemize}
\end{frame}




\begin{frame}
  \frametitle{Suggested reading}

  \begin{itemize}
  \item Jones (2009): \textbf{Chapter 5.4}
  \end{itemize}

\end{frame}



\end{document}
